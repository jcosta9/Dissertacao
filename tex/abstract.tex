Cyber-physical systems are a definite reality in our day-to-day lives, specially in the recent years. Nevertheless, the complexity inherent of these domains raises some challanges like unforeseen events, difficulties in depicting either the cyber or the physical processes, and the incomplete knowledge of the environment contexts, for example, might make the CPS unreliable at runtime, which could have disastrous effects. 

Seeking inspiration in processes from other fields is a very common activity in Computer Science. Nature, especially biology, has long served as a fruitfull source of methodologies like artificial intelligence approaches. The Negative Selection Algorithm, for example, is an immuno-based technique with multiple successful applications in the field of CPS, primarily in the field of fault diagnostics for the identification of anomalous behavior. The algorithm's explainability may bring expressive benefits for the design and verification of CPS by helping understand the property violation patterns, and thus enhance the system specification.

In this work, we propose a methodology that aims at increasing the reliability of CPSs. This is achieved by a systematical diagnosis of system properties violations based on data generated by a protoype, performed in the early stages of development. An immuno-inspired algorithm called Negative Selection (NSA) serves as an analytical redundancy method to isolate and identify the cause for property violation in the system. We believe that, by reasoning about why the property violations happen, the system specification and the property themselves may be refined, fault-tolerant mechanisms may be added, and, thus, safer and better applications might be written.