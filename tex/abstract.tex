Recently, Cyber-physical systems are becoming more relevant in our day-to-day lives, thus demanding higher reliability. The Runtime Verification field addresses this by relying on runtime monitors to verify the satisfiability of formalized system properties. When a violation is spotted, the system can alert the user, or run some predefined process in order to maintain the regular operation. Afterwards, the analyst can take a closer look and perform the proper correction measures.

Nevertheless, the complexity inherent of these domains raises some challenges like unforeseen events, difficulties in depicting either the cyber or the physical processes, and the incomplete knowledge of the environment contexts, for example. All of that may cause the monitors to misbehave or miss out on some important aspects that should be considered when monitoring a property.

Seeking inspiration in processes from other fields is a very common activity in Computer Science. The Negative Selection Algorithm, for example, is an immuno-based technique with multiple successful applications in the field of CPS, primarily in the field of fault diagnostics for the identification of anomalous behavior. The algorithm's explainability may bring expressive benefits for the design and verification of CPS by helping understand the property violation patterns, and thus enhance the system verification.

In this work, we propose a methodology that aims at increasing the reliability of CPSs. This is achieved by a systematical diagnosis of system properties violations based on data generated by a prototype, performed in the early stages of development. An immuno-inspired algorithm called Negative Selection (NSA) serves as an analytical redundancy method to isolate and identify the cause for property violation in the system. We believe that, by reasoning about why the property violations happen, the runtime monitors may be refined, fault-tolerant mechanisms may be added, and, thus, safer and better applications might be written.