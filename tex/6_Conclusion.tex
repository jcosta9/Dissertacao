Complex types uncertainties are a challenging issue in the area of self-adaptive system engineering whose solution can not be postponed anymore. Among these complex types, there is the model structure uncertainty, which is caused by a misalignment between the managed system model and the actual managed system. Since self-adaptive systems are continually adapting and evolving, such misalignment is frequently observed. Consequently, the controller is prone to make decisions based on outdated representations of the true system. Motivated by this problem, in this work we proposed two variations of the well known methods \textit{model averaging} and \textit{the model discrepancy}, used to deal with structural uncertainty. Supported by some information theory concepts and an unsupervised learning method, we aim at identifying, quantifying, and mitigating the model structure uncertainty in the realm of self-adaptive systems. We will evaluate our approach in a toy self-adaptive system and in the Body Sensor Network (BSN) case study, and report the results in a Goal-Question-Metric (GQM) fashion.

With this work we intend to, instead of tackling sources of model structure uncertainty individually, deal with them in a broader sense. We hope to raise the community's aware to the necessity of a unified and integrated view of structural uncertainty in self-adaptive systems. Different metrics other than entropy and mutual information can be adopted. Moreover, our framework allows the application of other learning techniques and heuristics to support the model averaging technique. Thus, for future work we plan to improve and the process through the introduction of new tools and more efficient techniques.